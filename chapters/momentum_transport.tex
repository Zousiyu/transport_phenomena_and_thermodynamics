\chapter{Momentum Transport}

\section{基本数学知识}
\subsection{Leibniz积分法则}

给出了一个对积分变量为函数的积分求导的法则,右侧第一项解释了积分随时间的变化,二三两项区域内物理量的得失。

\begin{equation}
    \frac{d}{dt} \int_{a(t)}^{b(t)} \phi(x,t) dx = \int_{a(t)}^{b(t)} \frac{\partial \phi}{\partial t} dx + \phi(b(t),t) \frac{\partial b}{\partial t} - \phi(a(t),t)\frac{\partial a}{\partial t}
\end{equation}

对由面$ S(t) $包围的封闭三维空间$ V(t) $来讲,假设面的移动速度为$ \mathbf{v_s} $,有如下积分法则:

\begin{equation}
    \frac{d}{dt} \int_{V(t)} \phi dV = \int_{V(t)} \frac{\partial \phi}{\partial t} dV + \int_{S(t)} \phi(\mathbf{v_s}\cdot \mathbf{n}) dS
\end{equation}

若控制体不移动,则:

\begin{equation}\label{Leibniz}
\frac{d}{dt} \int_{V(t)} \phi dV = \int_{V(t)} \frac{\partial \phi}{\partial t} dV
\end{equation}

\subsection{Reynolds输送定理}
Reynolds输送定理可用来定量地描述流场中流体性质的变化情形,譬如在一个控制容积(Control volume)之中含有某种流体。在经过一段时间之后,若控制容积中流体的总体性质B有所改变,则其变化必定是下列两种原因所造成的:

(1)总体性质B可能会因为本身的特性或外在因素的影响而产生随时间的变化,譬如流体中含有某种化学物质,该物质会因化学反应造成其质量B的变化,$ dB/dt $即为该物质质量的变化率。又譬如B为流体的动量,在受到外力的状况下,流体的动量会有所改变;

(2)因为流体流动所造成的变化:当流体流经控制容积时,会带入或带出一部份的物质。当流出控制容积的总体性质大于流入量,则净流出量为正,会造成控制容积之中该总体性质的减少;反之,流入量大于流出量便会造成该总体性质的增加。


对固定控制体来讲:

\begin{equation}
    \left( \frac{dB}{dt} \right)_{MV} = \int_V \frac{\partial}{\partial t}(b\rho)dV + \int_S b\rho\mathbf{v\cdot n} dS
\end{equation}

运用Gauss散度定理简化形式:

\begin{equation}
    \left( \frac{dB}{dt} \right)_{MV} = \int_V \left[ \frac{\partial}{\partial t}(\rho b) + \rho b \nabla\cdot(\mathbf{v}) \right]
\end{equation}

\section{无量纲数}

Reynolds Number,定义为惯性力和粘性力的比:

\[ Re = \frac{\rho UL}{\mu} \]

Grashof Number,定义为浮力和粘性力的比,$ \upsilon $为动力粘度:

\[ Gr = \frac{g\beta\Delta TL^3}{\upsilon^2} \]

Prandtl Number,定义为动量扩散率和热扩散率之比(也代表流动边界层和热边界层之比),$ k $为导热系数,$ \alpha $为热扩散系数,$ \upsilon $为动力粘度。随着$ Pr $增大,热对流的增大速率逐渐比热传导大:

\[ Pr = \frac{\mu c_p}{k} = \frac{\mu/\rho}{k/\rho c_p} = \frac{\upsilon}{\alpha} \]

P\'{e}clet Number,定义为对流传递速率与扩散传递速率之比,$ D $为质量扩散系数:

\[ Pe =
\begin{cases}
\frac{\rho ULc_p}{k} = \frac{UL}{\alpha} = RePr & \text{for heat transfer}\\
\frac{UL}{D} = ReSc & \text{for mass transfer}
\end{cases}
\]

Schmidt Number,定义为动量扩散率和质量扩散率之比:

\[ Sc = \frac{\upsilon}{D} \]

Nusselt Number,定义为对流和热传导之比:

\[Nu = \frac{hL}{k}\]

Mach Number,定义为速度和介质中声速之比:

\[ M = \frac{|\mathbf{v}|}{a} \]

声速由下式计算,其中$ \gamma=c_p/c_v $为比气体常数:

\[a=\sqrt{\gamma\left( \frac{\partial p}{\partial \rho} \right)_T}\]

对理想气体而言,简化如下:

\[a=\sqrt{\gamma RT}\]

Eckert Number,动能和焓之比。如果$ Ec\ll 1 $,能量方程中的粘性耗散和压力功就可以忽略:

\[Ec=\frac{\mathbf{v\cdot v}}{c_p\Delta T}\]

Weber Number,定义为惯性力和表面张力之比:

\[We=\frac{\rho U^2 L}{\sigma}\]

\section{随体导数}
微元中的物理量$ \phi $是位置和时间的函数,单位质量的物理量的变化率$ D\phi/Dt $定义为:

\begin{equation}
\frac{D\phi}{Dt} = \frac{\partial \phi}{\partial t}+\frac{\partial \phi}{\partial x}\frac{dx}{dt} + \frac{\partial \phi}{\partial y}\frac{dy}{dt} + \frac{\partial \phi}{\partial z}\frac{dz}{dt}=\frac{\partial \phi}{\partial t}+u\frac{\partial \phi}{\partial x}+v\frac{\partial \phi}{\partial y}+w\frac{\partial \phi}{\partial z}=\frac{\partial \phi}{\partial t}+\bm{u}\cdot \nabla \phi
\end{equation}

守恒方程的通用形式为:

\begin{equation}
\frac{\partial \rho\phi}{\partial t}+\nabla\cdot(\rho\phi\bm{u}) = 0
\end{equation}

\section{连续性方程}
由随体导数很容易推导关于物理量密度$ \rho $的守恒方程:

\begin{gather}
\frac{\partial \rho}{\partial t} + \frac{\partial \rho u}{\partial x} + \frac{\partial \rho v}{\partial y} + \frac{\partial \rho w}{\partial z} = 0 \\
\frac{\partial \rho}{\partial t} + \nabla \cdot (\rho\mathbf{U}) = 0
\end{gather}

\section{动量传递}
由随体导数很容易推导关于物理量速度$ \bm{u} $的守恒方程,其中$ \tau $为应力张量,$ F $为体积力:

\begin{equation}
\rho \frac{\partial \bm{u}}{\partial t} + \rho\nabla\cdot(\bm{uu}) = -\nabla p + \tau + F
\end{equation}

动量方程中主要包括两种力,表面力(压力、粘性力)和体积力(重力等)。

\section{带扩散的物理量的一般守恒方程}

增加率+净流出=扩散增加率+源项

\begin{equation}
\frac{\partial (\rho\phi)}{\partial t} + \nabla(\rho\phi\bm{u}) = \nabla(\Gamma\nabla\phi) + S
\end{equation}

在有限体积法中,会在整个控制体内对方程积分:

\begin{equation}
\int_{CV}\frac{\partial (\rho\phi)}{\partial t}dV + \int_{CV}\nabla(\rho\phi\bm{u})dV = \int_{CV}\nabla(\Gamma\nabla\phi)dV + \int_{CV}SdV
\end{equation}

然后用Gauss散度定理将对流-扩散项从体积分改为面积分,同时用\autoref{Leibniz}所示的Leibniz积分法则改写第一项。

\begin{equation}
\frac{\partial}{\partial t}\left(\int_{CV}\rho\phi dV\right) + \int_{A}\bm{n}\cdot(\rho\phi\bm{u})dV = \int_{A}\bm{n}\cdot(\Gamma\nabla\phi)dV + \int_{CV}SdV
\end{equation}

有限体积法的积分方程具有非常自然的物理意义,左边第一项表示物理量$ \phi $在控制体中的变化率,坐标第二项表示控制体面上的净流出量(即对流项),右边第一项表示扩散引起的增量,最后一项为源项。

对于稳态问题,时间导数为0:

\begin{equation}
\int_{A}\bm{n}\cdot(\rho\phi\bm{u})dV = \int_{A}\bm{n}\cdot(\Gamma\nabla\phi)dV + \int_{CV}SdV
\end{equation}

对瞬态问题,需要在每一个时间步内进行积分:

\begin{equation}
\int_{\Delta t}\frac{\partial}{\partial t}\left(\int_{CV}\rho\phi dV\right) + \int_{\Delta t}\int_{A}\bm{n}\cdot(\rho\phi\bm{u})dV = \int_{\Delta t}\int_{A}\bm{n}\cdot(\Gamma\nabla\phi)dV + \int_{\Delta t}\int_{CV}SdV
\end{equation}

\section{自然对流 natural convection}

自然对流控制方程:


在强制对流里面,使用$Re$数来描述流动行为,其表示惯性力(the inertial
forces)与粘性力(the viscous forces)之比。对于内部驱动的流动行为——比如自然对流,初始速度未知,无法使用$Re$数表征流动行为。通常用The Grashof number来描述自然对流,其表示流体浮力与粘性力的比值(It describes the ratio of the time scales for viscous diffusion in the fluid and the internal driving force (the buoyancy force).)。当$Gr>10^9$时,自然对流转变为湍流。

对湿空气来讲,密度依赖于温度和含水率,$Gr$数定义如下:
\begin{equation}
    Gr = \frac{g\rho (\rho_{ext} - \rho_s) L^3}{\mu^2}
\end{equation}

$\rho_s$代表热表面的密度(solid),$\rho_{ext}$代表the free stream density,$\rho$代表流体密度。

对干空气来讲,密度仅依赖于温度,$Gr$数定义如下:
\begin{equation}
    Gr = \frac{g\beta (T_s - T_{ext})L^3}{(\mu/\rho)^2}
\end{equation}

通常,$\mu/\rho=\upsilon$,定义为动力粘度。

$\beta$为热膨胀系数(the coefficient of thermal expansion),定义为:
\begin{equation}
    \beta = -\frac{1}{\rho} \left( \frac{\partial \rho}{\partial T} \right)_T
\end{equation}

对理想气体而言,$\beta$简化为:
\begin{equation}
    \beta = \frac{1}{T}
\end{equation}

同样,The Rayleigh number也可以用来表征自然对流,$Ra$数定义如下:
\begin{equation}
    Ra = GrPr =\frac{g\beta\rho^2 C_p |T-T_{ext}|L^3}{k\mu}
\end{equation}

其中,$Pr$为The Prandtl number,定义为:
\begin{equation}
    Pr = \frac{\upsilon}{\alpha} = \frac{c_p\mu}{k}
\end{equation}

$\alpha$为热扩散系数,定义为:
\begin{equation}
    \alpha = \frac{k}{\rho c_p}
\end{equation}


若密度仅仅依赖于温度,$Ra$数定义如下:
\begin{equation}
    Ra = \frac{g\rho C_p |\rho_{ext}-\rho_s|L^3}{k\mu}
\end{equation}
