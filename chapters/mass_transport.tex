\chapter{Mass Transport}

\section{浓物质传递}
由$i$个物质构成的,有$j$个反应的反应流、浓物质系统的质量守恒方程为:

\begin{equation}
    \frac{\partial}{\partial t}(\rho\omega_i) + \nabla\cdot(\rho\omega_i\bm{u}) = -\nabla\cdot\bm{j}_i + R_i
\end{equation}

其中$\omega_i$表示$i$物质的质量分数,$\bm{j}_i$为质量通量(mass flux)源项,表示分子扩散、电场迁移、热扩散等引起的通量。

\subsection{Maxwell-Stefan Description}
\textbf{Maxwell-Stefan}是最详细的扩散模型,计算消耗最大,是一个扩散占主导的模型。多组分混合物中,相对于质量平均速度的质量通量可使用通用的Fick方程定义,总的扩散通量依赖于物质的浓度梯度、温度、压力和外部驱动力:

\begin{equation}
    \bm{j}_i = -\rho\omega_i\sum_{k=1}^{Q} \tilde{D}_{ik} \bm{d}_{k} - D_i^T\nabla\ln T
\end{equation}

$\tilde{D}_{ik}$是多组分Fick扩散系数,$D_i^T$是热扩散系数,$\bm{d}_k$是扩散驱动力。对理想气体而言,扩散驱动力可表示为:

\begin{equation}
    \bm{d}_k = \frac{1}{cR_gT} \left[ \nabla p_k-\omega_k\nabla p - \rho_k \bm{g}_k +\omega_k \sum_{l=1}^{Q} \rho_l \bm{g}_l \right]
\end{equation}

其中,$p_k$为气体分压,$\rho_k$为物质$k$的密度,$\bm{g}_k$为作用于物质$k$的外部作用力(例如电场)。

\subsection{Mixture-Averaged Approximation}
\textbf{Mixture-Averaged}模型假定分压和温度变化对多组分扩散的影响可以忽略,假定由分子扩散引起的质量通量符合Fick定律,则分子扩散通量正比于摩尔分数的梯度,定义为:

\begin{gather}
    \bm{j}_{md,i} = -\rho_i D_i^m \frac{\nabla x_i}{x_i} \\
    \bm{j}_{md,i} = -\left( \rho D_i^m \nabla \omega_i + \rho \omega_i D_i^m \frac{\nabla M}{M} \right) \\
    \rho_i = \rho\omega_i, x_i = \frac{\omega_i}{M_i} M
\end{gather}

其中,混合物平均扩散系数定义为:

\begin{equation}
    D_i^m = \frac{1-\omega_i}{\sum_{k\neq i}^N \frac{x_k}{D_{ik}}}
\end{equation}

\subsection{Fick's law}
\textbf{Fick's law}是一个通用的模型,通常用来描述分子扩散不占主导地位的系统,不需要多组分扩散系数,计算消耗最低。

\subsection{热扩散 Thermal Diffusion}

由温差引起的扩散称为\textbf{Soret效应},当混合物温差大、分子质量相差大时会产生热扩散,分子质量高的物质在低温区域积累,分子质量低的物质在高温处积累。多组分混合物所有热扩散系数为0:

\begin{equation}
    \sum_{i=1}^Q D_i^T = 0
\end{equation}

\section{多组分扩散系数}
Maxwell-Stefan扩散系数$ D_{ik} $,Fick扩散系数$ D_w^F $,热扩散系数$ D_w^T $。

二元Fick扩散系数是对称的(symmetric),其和Maxwell-Stefan扩散系数$D_{ik}$可以相互转换。

\begin{equation}
    \tilde{D}_{ik} = \tilde{D}_{ki}
\end{equation}

\section{化学吸附}
化学吸附只能发生在固体表面的活性位点$ \sigma $上,定义被A覆盖的活性位点与总活性位点之比为吸附率,剩下的为空位率:

\begin{gather}
    \theta_A = \frac{\sigma_A}{\sigma} \\
    \theta_V = 1-\theta_A
\end{gather}

分子的吸附速率$ r_{ads} $与表面空位率和气体分压成正比,解吸率与表面占有率成正比:

\begin{gather}
    r_{ads} = k_{ads}p_A\theta_V \\
    r_{des} = k_{des}\theta_A
\end{gather}

达到吸附平衡时,表观速率$ r=r_{ads}-r_{des}=0 $,有:

\[k_{ads}p_A(1-\theta_A)=k_{des}\theta_A\]

令$ K_A=\frac{k_{ads}}{k_{des}} $,称为\textbf{吸附平衡常数},得\autoref{Langmuir},称为Langmuir吸附等温式:

\begin{equation}\label{Langmuir}
    \theta_A = \frac{K_Ap_A}{1+K_Ap_A}
\end{equation}

为了根据本体浓度$ c $和表面浓度$ cs $来建立传递和反应方程,执行以下代换,其中$ \Gamma_s $为表面得最大吸附浓度:

\[\theta_A = c_s/\Gamma_s\]
\[p_A=cRT\]

将得到下式,粗体为吸附和脱附反应速率常数。

\begin{gather}
    r_{ads} = \bm{\frac{k_{ads}RT}{\Gamma_s}}c(\Gamma_s-c_s)\\
    r_{des} = \bm{\frac{k_{des}}{\Gamma_s}}c_s
\end{gather}

对于表面来讲,涉及表面扩散、吸附脱附、表面反应,质量守恒为:

\begin{equation}
    \frac{\partial c_s}{\partial t}+\nabla\cdot(-D_s\nabla c_s) = r_{ads}-r_{des}=R
\end{equation}

本体溶液的质量守恒用经典的对流扩散方程描述:

\begin{equation}
    \frac{\partial c}{\partial t} + \nabla\cdot(-D\nabla c+\bm{u}c) = R
\end{equation}
