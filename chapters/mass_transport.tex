\chapter{Mass Transport}

\section{多组分质量传递基本概念}

对多组分流体混合物而言$ m=\sum m_i $,质量浓度(\si{\kilogram\per\meter\cubed})定义为:

\[\rho_i = \frac{m_i}{V} \]

混合物的密度即为全部物质的质量浓度之和:

\[\rho = \sum\rho_i \]

类似地,混合物的物质的量(\si{\mole})、质量分数($ \sum \omega_i = 1 $)、摩尔分数分别定义为:

\[ n_i=\frac{m_i}{M_i} \]
\[ \omega_i = \frac{m_i}{m} \]
\[ x_i = \frac{n_i}{n} \]

并且有如下转换关系:

\[\rho_i = \rho\omega_i = \rho\frac{M_i}{M}x_i \]

混合物的等效摩尔质量(\si{\kilogram\per\mole})定义为:

\[ M = \sum M_i x_i \]

如果气体混合物是理想气体,则满足:

\[ PV=nRT \]
\[ PV=mR_m T \]

其中$ R_m $是混合物气体参数,定义为:

\[ R_m = \frac{n}{m}R = \frac{R}{M} \]

气体$ i $的分压为$ P_i $,根据Dalton定律有:

\[ P=\sum P_i \]

\section{Mass diffusivity}
扩散系数分为三类,热扩散系数$ \alpha $、质量扩散系数$ D $、动量扩散系数$ \upsilon $,单位均为\si{\square\meter\per\second}。定义Schmidt数为$ Sc=\upsilon/D $,根据动力学理论,理想气体的$ Sc=1 $,故$ Sc $数其主要用来表征气体与理想气体的偏差(即表征气体的非理想性)。

浓度梯度引起分子扩散,通常化合物在空气中的扩散系数为在水中的扩散系数的1000倍左右。扩散系数的国际单位为\si{\square\meter\per\second}。对气体而言,根据动力学理论可知,气体的扩散系数与温度和压力相关,关系通常为$ f(T^n/p) $(通常$ n=3/2 $),气体的扩散系数通常随温度升高或压力下降而增大。实际上,由于气体的非理想性,$ n $通常要更大,例如水汽在空气中扩散,$ n=2.072(250\si{\kelvin}<T<459\si{\kelvin}) $。

\textbf{气体的二元扩散系数}常用Chapman–Enskog theory,其中,$ M $为分子质量(\si{\kilogram\per\mol}),$ p $为压力(\si{\pascal}),$ \sigma_{ij} $为平均碰撞直径(\si{\angstrom}),$ \Omega $是一个温度相关的碰撞积分:

\begin{equation}
D_{AB} = 2.695\times 10^{-3}\cdot \frac{\sqrt{T^3 (M_A+M_B)/(2000M_AM_B)}}{p\sigma_A\sigma_B\Omega_D}
\end{equation}

\textbf{液体的二元扩散系数}常用Wilke-Chang equation,其中$ \mu_B $为动力粘度(\si{\newton\second\per\square\meter}),$ V_A $为正常沸点下的分子体积(\si{\cubic\meter\per\mole}),$ \phi_B $为溶剂的无量纲关联系数,默认为1:

\begin{equation}
D_{AB} = 3.5\times 10^{-15} \frac{(\phi_B M_B)^{1/2} T}{\mu V_A^{0.6}}
\end{equation}

几个扩散系数的参照:

\begin{table}[!htb]
    \centering
    \caption{几个扩散系数参照\si{\square\meter\per\second}}
    \begin{tabular}{cccc}
        \toprule
        物质 & 在空气中1atm & 在水中 \\
        \midrule
        甲醇 & $ 1.5\times10^{-5} $ & $ 1.64\times10^{-9} $ \\
        乙醇 & $ 1.1499\times10^{-5} $ & $ 1.216895\times10^{-9} $ \\
        氧气 & $ 1.76\times10^{-5}(25\si{\degreeCelsius}) $ & - \\
        苯酚 & $ 8.2\times10^{-6} $ & $ 9.1\times10^{-10} $ \\
        \bottomrule
    \end{tabular}
\end{table}

上面两个公式是基于分子动力学,比较难用,主要是涉及分子动力学的常数比较难寻找。Fuller提出了一个气体扩散系数的经验关联式,形如$ D_{AB}=f(T^{1.75}/p) $,简单易用,误差不超过$ 10\% $。详情参见文献\cite{poling2001properties}。

\begin{equation}
D_AB=\frac{0.00143T^{1.75}}{pM_{AB}^{1/2}\left( \sum_{V_A}^{1/3} + \sum_{V_B}^{1/3} \right)^2 }
\end{equation}

其中$ D_{AB} $为二元扩散系数(\si{cm\squared\per\second});$ T $为温度(\si{\kelvin}),$ M_{AB}=2M_A M_B/(M_A + M_B) $;$ M_A $,$ M_B $为相对分子质量(\si{\g\per\mole});$ p $为压力(\si{bar})。

\section{Thermal Diffusion/Soret effect}

由温差引起的扩散称为\textbf{Soret效应},也称热泳(\url{https://en.wikipedia.org/wiki/Thermophoresis})。当混合物温差大、分子质量相差大时会产生热扩散,分子质量高的物质在低温区域积累,分子质量低的物质在高温处积累。多组分混合物所有热扩散系数之和为0:

\begin{equation}
\sum_{i=1}^Q D_i^T = 0
\end{equation}

\section{浓物质传递}
由$i$个物质构成的,有$j$个反应的反应流、浓物质系统的质量守恒方程为:

\begin{equation}
    \frac{\partial}{\partial t}(\rho\omega_i) + \nabla\cdot(\rho\omega_i\bm{u}) = -\nabla\cdot\bm{j}_i + R_i
\end{equation}

其中$\omega_i$表示$i$物质的质量分数,$\bm{j}_i$为质量通量(mass flux)源项,表示分子扩散、电场迁移、热扩散等引起的通量。

\subsection{Maxwell-Stefan Description}
\textbf{Maxwell-Stefan}是最详细的扩散模型,计算消耗最大,是一个扩散占主导的模型。多组分混合物中,相对于质量平均速度的质量通量可使用通用的Fick方程定义,总的扩散通量依赖于物质的浓度梯度、温度、压力和外部驱动力:

\begin{equation}
    \bm{j}_i = -\rho\omega_i\sum_{k=1}^{Q} \tilde{D}_{ik} \bm{d}_{k} - D_i^T\nabla\ln T
\end{equation}

$\tilde{D}_{ik}$是多组分Fick扩散系数,$D_i^T$是热扩散系数,$\bm{d}_k$是扩散驱动力。对理想气体而言,扩散驱动力可表示为:

\begin{equation}
    \bm{d}_k = \frac{1}{cR_gT} \left[ \nabla p_k-\omega_k\nabla p - \rho_k \bm{g}_k +\omega_k \sum_{l=1}^{Q} \rho_l \bm{g}_l \right]
\end{equation}

其中,$p_k$为气体分压,$\rho_k$为物质$k$的密度,$\bm{g}_k$为作用于物质$k$的外部作用力(例如电场)。

\subsection{Mixture-Averaged Approximation}
\textbf{Mixture-Averaged}模型假定分压和温度变化对多组分扩散的影响可以忽略,假定由分子扩散引起的质量通量符合Fick定律,则分子扩散通量正比于摩尔分数的梯度,定义为:

\begin{gather}
    \bm{j}_{md,i} = -\rho_i D_i^m \frac{\nabla x_i}{x_i} \\
    \bm{j}_{md,i} = -\left( \rho D_i^m \nabla \omega_i + \rho \omega_i D_i^m \frac{\nabla M}{M} \right) \\
    \rho_i = \rho\omega_i, x_i = \frac{\omega_i}{M_i} M
\end{gather}

其中,混合物平均扩散系数定义为:

\begin{equation}
    D_i^m = \frac{1-\omega_i}{\sum_{k\neq i}^N \frac{x_k}{D_{ik}}}
\end{equation}

\subsection{Fick's law}
\textbf{Fick's law}是一个通用的模型,通常用来描述分子扩散不占主导地位的系统,不需要多组分扩散系数,计算消耗最低。

\section{多组分扩散系数}
Maxwell-Stefan扩散系数$ D_{ik} $,Fick扩散系数$ D_w^F $,热扩散系数$ D_w^T $。

二元Fick扩散系数是对称的(symmetric),其和Maxwell-Stefan扩散系数$D_{ik}$可以相互转换。

\begin{equation}
    \tilde{D}_{ik} = \tilde{D}_{ki}
\end{equation}

\section{化学吸附}
化学吸附只能发生在固体表面的活性位点$ \sigma $上,定义被A覆盖的活性位点与总活性位点之比为吸附率,剩下的为空位率:

\begin{gather}
    \theta_A = \frac{\sigma_A}{\sigma} \\
    \theta_V = 1-\theta_A
\end{gather}

分子的吸附速率$ r_{ads} $与表面空位率和气体分压成正比,解吸率与表面占有率成正比:

\begin{gather}
    r_{ads} = k_{ads}p_A\theta_V \\
    r_{des} = k_{des}\theta_A
\end{gather}

达到吸附平衡时,表观速率$ r=r_{ads}-r_{des}=0 $,有:

\[k_{ads}p_A(1-\theta_A)=k_{des}\theta_A\]

令$ K_A=\frac{k_{ads}}{k_{des}} $,称为\textbf{吸附平衡常数},得\autoref{Langmuir},称为Langmuir吸附等温式:

\begin{equation}\label{Langmuir}
    \theta_A = \frac{K_Ap_A}{1+K_Ap_A}
\end{equation}

为了根据本体浓度$ c $和表面浓度$ cs $来建立传递和反应方程,执行以下代换,其中$ \Gamma_s $为表面得最大吸附浓度:

\[\theta_A = c_s/\Gamma_s\]
\[p_A=cRT\]

将得到下式,粗体为吸附和脱附反应速率常数。

\begin{gather}
    r_{ads} = \bm{\frac{k_{ads}RT}{\Gamma_s}}c(\Gamma_s-c_s)\\
    r_{des} = \bm{\frac{k_{des}}{\Gamma_s}}c_s
\end{gather}

对于表面来讲,涉及表面扩散、吸附脱附、表面反应,质量守恒为:

\begin{equation}
    \frac{\partial c_s}{\partial t}+\nabla\cdot(-D_s\nabla c_s) = r_{ads}-r_{des}=R
\end{equation}

本体溶液的质量守恒用经典的对流扩散方程描述:

\begin{equation}
    \frac{\partial c}{\partial t} + \nabla\cdot(-D\nabla c+\bm{u}c) = R
\end{equation}

\section{多孔介质质量传递}

对每一个组分应用质量守恒定律,并忽略源汇项,得出基于质量浓度的\textbf{单组分质量守恒方程}:

\begin{equation}
\frac{\partial \rho_i}{\partial t} + \nabla\cdot(\rho_i\bm{U_i}) = 0
\end{equation}

将方程中的质量浓度和内在平均速度汇总,得出全部组分的质量守恒方程:

\begin{equation}
\frac{\partial \rho}{\partial t} + \nabla\cdot(\sum\rho_i\bm{U_i}) = 0
\end{equation}

使用$ \bm{U}=\frac{1}{\rho}\sum\rho_i\bm{U_i} $表示质量平均速度,得出:

\begin{equation}
\frac{\partial \rho}{\partial t} + \nabla\cdot(\rho\bm{U}) = 0
\end{equation}

我们定义物质$ i $相对于质量平均速度的运动为扩散(diffusion),$ \bm{U_i}-\bm{U} $为扩散速度,则扩散通量为:

\begin{equation}
\bm{j_i} = \rho_i(\bm{U_i} - \bm{U})
\end{equation}

则,考虑对流-扩散的单组分质量守恒方程为:

\begin{equation}
\frac{\partial \rho_i}{\partial t} + \nabla\cdot(\rho_i\bm{U}) = -\nabla\cdot \bm{j_i}
\end{equation}

根据Fick定律,$ \bm{j_i} = -D\nabla\rho $,在上式两端同时乘上孔隙率$ \varepsilon $以考虑多孔介质内的质量传递:

\begin{equation}
\varepsilon\frac{\partial \rho_i}{\partial t} + \varepsilon\nabla\cdot(\rho_i\bm{U}) = \varepsilon\nabla\cdot(D\nabla \rho_i)
\end{equation}

运用Dupuit–Forchheimer relationship,考虑多孔介质内部的有效扩散系数,并将上式改写为关于摩尔浓度:

\begin{equation}
\varepsilon\frac{\partial c_i}{\partial t} + \varepsilon\nabla\cdot(c_i\bm{u}) = \varepsilon\nabla\cdot(D_{eff}\nabla c_i)
\end{equation}

\section{化学反应}

化学反应速率表达式一般为:

\begin{equation}
\frac{dc}{dt} = -kc^n
\end{equation}

反应速率常数常常于温度和和活化能相关,一般由Arrhenius relationship给出:

\begin{equation}
k = A \exp(-\frac{E}{RT})
\end{equation}

和温度相关的Arrhenius表达式一般为:

\begin{equation}
k = A \left( \frac{T}{T_{ref}} \right) \exp(-\frac{E}{RT})
\end{equation}

