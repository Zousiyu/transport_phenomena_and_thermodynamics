\chapter{Numerical Analysis}

\section{数值微分和数值积分}
当函数是初等函数时,可以使用\textbf{符号计算}求出其微积分,一旦函数是非初等函数,就只能使用\textbf{数值计算}方法了。

\subsection{Taylor's Formula}
首先看泰勒定理,其中$ c\in(x,x+h) $:
\begin{equation}
f(x) =f(x_0)+f'(x_0)(x-x_0) + \frac{f''(x_0)}{2!}(x-x_0)^2 +
\underbrace{\frac{f^{(n)}(c)}{n!}(x-x_0)^n}_{\text{remainder term}}
\end{equation}

\subsection{有限差分}
首先看导数的定义:

\begin{equation}
f'(x) = \lim_{h\rightarrow 0} \frac{f(x+h)-f(x)}{h}
\end{equation}

如果$ f $二阶连续可微,泰勒展开为:

\begin{equation}
f(x+h) = f(x)+hf'(x)+\frac{h^2}{2}f''(c)
\end{equation}

得出大名鼎鼎的\textbf{两点前向差分公式},其中$ c\in(x,x+h) $:

\begin{equation}
f'(x) = \frac{f(x+h)-f(x)}{h} - \frac{h}{2}f''(c)
\end{equation}

\section{Newton Cotes数值积分}
对定义在区间$ [a,b] $上的函数$ f $,可计算通过函数$ f(x) $的一些点的插值多项式的积分来近似$ f(x) $的积分。

梯形法则,其中$ h=x_1-x_0 $,$ c\in(x_0,x_1) $:

\begin{equation}
\int_{x_0}^{x_1}f(x)dx = \frac{h}{2}(y_0 + y_1) - \frac{h^3}{12}f''(c)
\end{equation}

辛普森法则,其中$ h=x_2-x_1=x_1-x_0 $,$ c\in(x_0,x_2) $:

\begin{equation}
\int_{x_0}^{x_1}f(x)dx = \frac{h}{3}(y_0 +4 y_1+y_2) - \frac{h^5}{12}f^{(4)}(c)
\end{equation}

\section{Gauss数值积分}

\begin{equation}
\int_1^1f(x)dx \approx \sum_{i=1}^n c_i f(x_i)
\end{equation}