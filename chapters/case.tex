\chapter{实例}

整理几个上传一反的应用实例。

\section{固定床}

研究单个处于热流中的球形催化剂颗粒,如甲烷蒸汽重整或水汽变换。

\subsection{外扩散}

流体流过固体催化剂时,在固体表面会产生一个边界层。由于边界层内部速度几乎为0(壁面处等于0),所以穿过边界层到固体表面的质量传递是分子扩散引起的,这层边界层内的传递阻力也是流体向固体表面进行质量传递和能量传递的最大阻力。传质边界层$ \delta $一般定义为从固体表面到物质浓度达到流体主体浓度的99\%的距离,固体催化剂外表面的边界层传递阻力称为外扩散阻力。实际上,由于固体几何形貌的差异,流体流过固体时产生的边界层厚度不一致,造成固体表面的温度和浓度不均一。处理这些问题时,一般假设边界层厚度处处相等,这样可以将相间传递问题简化为一维问题,使复杂问题大大简化且能保持足够的精度。

由于流体速度越大,边界层厚度越小,所以其传递阻力主要由床层的水力学性能决定,除此之外,固体颗粒的几何性质、尺寸以及流体的物理属性等都会对传递阻力造成很大的影响。流体向固体表面的传质通量由传质系数和边界层两侧物理量的梯度来表示,传递系数反应了传递阻力的大小,传递阻力越大,传递系数越小。

\[ N=kA(c-c_s) \]

其中,$ N $为传质通量(\si{\mole\per\meter\cubed}),$ k $为传质系数(\si{\meter\per\second}),$ A $为固体的表面积(\si{\meter\squared})。

传质系数包含在Sherwood数中,定义为,

\[ Sh = \frac{hL}{D} = f(Re, Sc)\]

用$ j $因子关联气固传质和传热实验数据,传质和传热的$ j $因子分别定义为,

\[ j_m = \frac{h_m\rho}{G}Sc^{2/3}=\frac{h_m}{u}Sc^{2/3}=StSc^{2/3} \]

\[ j_h = \frac{h_h}{Gc_p}Pr^{2/3} =StPr^{2/3}\]


其中,传质系数$ h_m $(\si{\meter\per\second}),传热系数$ h_h $(\si{\watt\per\meter\squared\per\second})。传质系数将随着质量速度的增大而增大,即较高的流速加快了外扩散的传质速率,较低的流速会使外扩散阻力变大,甚至会成为过程的控制步骤。

将$ St $数的表达式带入$ j $因子表达式,得,

\[ Sh = j_m ReSc^{1/3} \]

\[ Nu = j_h RePr^{1/3} \]

传质关联由大量的实验数据得来,通常,对球形颗粒的传质关联为:

\[ Sh= 2+0.6Re^{1/2}Sc^{1/3} \]

球形颗粒的传热关联与传质关联形式完全一致,

\[ Nu=2+0.6Re^{1/2}Pr^{1/3} \]



\subsection{内扩散}

\subsection{多孔介质性质}

\subsection{1D homogeneous model}

\subsection{1D heterogeneous model}